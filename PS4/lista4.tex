\documentclass[10pt]{article}
\usepackage[margin=2cm]{geometry}
\usepackage{lipsum}
\usepackage{amsmath}
\usepackage[inline]{enumitem}
\usepackage{verbatim}
\usepackage{fancyvrb}
\usepackage{pgfornament}
\usepackage{amssymb, tabularx, xcolor, nccmath}
\usepackage[utf8]{inputenc}
\usepackage{cancel}
\usepackage[portuguese]{babel}
\usepackage{bbm}
\usepackage{titlesec}
\usepackage{setspace, mathtools}
\usepackage{multirow}

\titleformat*{\section}{\sffamily\large\bfseries}
\titleformat*{\subsection}{\sffamily\normalsize\bfseries}
\makeatletter
\renewcommand*\env@matrix[1][\arraystretch]{%
  \edef\arraystretch{#1}%
  \hskip -\arraycolsep
  \let\@ifnextchar\new@ifnextchar
  \array{*\c@MaxMatrixCols c}}
\makeatother

\usepackage{hyperref}
\hypersetup{
    colorlinks=true,
    linkcolor=magenta,
    filecolor=magenta,      
    urlcolor=magenta,
}

\usepackage[T1]{fontenc}
\usepackage{ccfonts}
\renewcommand{\bfseries}{\fontfamily{cmbr}\fontseries{bx}\selectfont}
\DeclareMathAlphabet{\mathbf} {OT1}{cmbr}{bx}{n}
\DeclareMathAlphabet{\mathbold}{OML}{cmbrm}{b}{it}

\setlength{\parindent}{4em}
\setlength{\parskip}{1em}
\renewcommand{\baselinestretch}{1.2}


\def\changemargin#1#2{\list{}{\rightmargin#2\leftmargin#1}\item[]}
\let\endchangemargin=\endlist 



\DeclareMathOperator*{\argmin}{arg\,min}

\newcommand*{\QEDA}{\hfill\ensuremath{\blacksquare}}%
\newcommand*{\QEDB}{\hfill\ensuremath{\square}}%
\DeclareMathOperator*{\plim}{plim}

\setlength\parindent{0pt}

\usepackage{dsfont}

\newcommand\Z{\mathbf{Z}}
\newcommand\E{\mathbb{E}}

\newcommand\R{\mathbb{R}}
\newcommand\D{\mathbf{D}}
\newcommand\C{\mathbf{D}}
\newcommand\X{\mathbf{X}}
\newcommand\yy{\mathbf{y}}
\newcommand\ii{{\boldsymbol{\iota}}}
\renewcommand\O{{\boldsymbol{\Omega}}}
\newcommand\y{\mathbf{y}}
\newcommand\e{\mathbf{e}}
\renewcommand\u{\mathbf{u}}
\renewcommand\d{\mathbf{d}}
\newcommand\resid{\mathbf{u}}
\newcommand\0{\mathbf{0}}
\newcommand\betabold{\pmb{\beta}}
\newcommand\ones{\boldsymbol{\iota}}
\newcommand\epbold{\boldsymbol{\epsilon}}
\newcommand\varep{\boldsymbol{\varepsilon}}
\newcommand\x{\mathbf{x}}
\newcommand\z{\mathbf{z}}
\newcommand\Q{\mathbf{Q}}
\newcommand\I{\mathbf{I}}
\newcommand\w{{\mathbf{w}}}
\newcommand\wm{\bar{\mathbf{w}}}
\newcommand\M{{\mathbf{M}}}
\renewcommand\P{{\mathbf{P}}}
\newcommand\var{\operatorname{var}}
\newcommand\cov{\operatorname{cov}}
\renewcommand\b{\mathbf{b}}
\renewcommand\e{\mathbf{b}}


\usepackage{amssymb,graphicx}
\def\tallqed{\smash{\scalebox{.75}[1.025]{\color{blue!50!black}$\blacksquare$}}}

\tolerance=1
\emergencystretch=\maxdimen
\hyphenpenalty=10000
\hbadness=10000

\newcommand{\prnt}[1]{\ensuremath{\left(#1\right)}} %parentheses
\newcommand{\colch}[1]{\ensuremath{\left[#1\right]}} %square brackets
\newcommand{\chave}[1]{\ensuremath{\left\{#1\right\}}}  %curly brackets

\usepackage{stackengine}
\renewcommand\useanchorwidth{T}
\usepackage{graphicx}
\stackMath

\allowdisplaybreaks

\usepackage{MnSymbol}

\newcommand{\mytext}[1]% #1 = same as intertext
{&\parbox{0.94\textwidth}{\rule{0pt}{.5\baselineskip}\\
\textrm{#1}\\
\rule{0pt}{.5\baselineskip}}&\\}

\newcounter{exercise}
\newcounter{problem}[exercise]
\newcommand{\myitem}{\stepcounter{problem}\tag*{\alph{problem})}}

\usepackage{amsmath, amsthm, amssymb, amsfonts, enumitem, fancyhdr, color, comment, graphicx, environ, csquotes}


\newenvironment{problem}[2][Problem]{\begin{trivlist}
\item[\hskip \labelsep {\bfseries #1}\hskip \labelsep {\bfseries #2.}]}{\end{trivlist}}
\newenvironment{sol}
    {\\[1em] {\color{magenta}\text{Resposta.}}
    }
    {{\color{blue!50!black}\QEDA}}

\setlength{\parskip}{\baselineskip}%

\usepackage[framed,numbered,useliterate]{mcode}
\usepackage{listings}
\lstset{
    inputencoding = utf8,  % Input encoding
    extendedchars = true,  % Extended ASCII
    texcl         = true,  % Activate LaTeX commands in comments
    captionpos    = b,     % Caption position
    literate      =        % Support additional characters
      {á}{{\'a}}1  {é}{{\'e}}1  {í}{{\'i}}1 {ó}{{\'o}}1  {ú}{{\'u}}1
      {Á}{{\'A}}1  {É}{{\'E}}1  {Í}{{\'I}}1 {Ó}{{\'O}}1  {Ú}{{\'U}}1
      {à}{{\`a}}1  {è}{{\`e}}1  {ì}{{\`i}}1 {ò}{{\`o}}1  {ù}{{\`u}}1
      {À}{{\`A}}1  {È}{{\'E}}1  {Ì}{{\`I}}1 {Ò}{{\`O}}1  {Ù}{{\`U}}1
      {ä}{{\"a}}1  {ë}{{\"e}}1  {ï}{{\"i}}1 {ö}{{\"o}}1  {ü}{{\"u}}1
      {Ä}{{\"A}}1  {Ë}{{\"E}}1  {Ï}{{\"I}}1 {Ö}{{\"O}}1  {Ü}{{\"U}}1
      {â}{{\^a}}1  {ê}{{\^e}}1  {î}{{\^i}}1 {ô}{{\^o}}1  {û}{{\^u}}1
      {Â}{{\^A}}1  {Ê}{{\^E}}1  {Î}{{\^I}}1 {Ô}{{\^O}}1  {Û}{{\^U}}1
      {œ}{{\oe}}1  {Œ}{{\OE}}1  {æ}{{\ae}}1 {Æ}{{\AE}}1  {ß}{{\ss}}1
      {ç}{{\c c}}1 {Ç}{{\c C}}1 {ø}{{\o}}1  {å}{{\r a}}1 {Å}{{\r A}}1
      {ñ}{{\~n}}1  {Ñ}{{\~N}}1  {¿}{{?`}}1  {¡}{{!`}}1
      % ¿ and ¡ are not correctly displayed if inconsolata font is used
      % together with the lstlisting environment. Consider typing code in
      % external files and using \lstinputlisting to display them instead.      
  }
    
\begin{document}

\stepcounter{exercise}
\newdimen\headerwidth


\begin{center}
  \framebox{
    \vbox{
      \headerwidth=\textwidth
      \vspace{1mm}
      \advance\headerwidth by -0.22in
      \hbox to \headerwidth {\it Doutorado em Economia - EPGE/FGV \hfill MDPEMF024 - Métodos Numéricos}
      \vspace{5mm}
      \hbox to \headerwidth {{\Large \hfill Lista \#4 \hfill}}
      \vspace{6mm}
      \hbox to \headerwidth {\hfill \today \hfill}
      \vspace{5mm}
      \hbox to \headerwidth {{\it Aluno: Rafael Vetromille  \hfill Professor: Cézar Santos / TA: Ana Paula Ruhe}}
      \vspace{1mm}
      }
    }
\end{center}

This question will ask you to solve numerically a heterogeneou-agent economy. Suppose there is a continuum of individuals that are subject to endowment shocks. A person's endowment is $e^z$, where $z$ follows the following stochastic process:
\begin{align*}
z' = \rho z  + \varepsilon
\end{align*}
where $\varepsilon \sim N(0, \sigma^2)$. The individual's instantaneous utility function is given by
\begin{align*}
u(c) = \frac{c^{1-\gamma}- 1}{1 - \gamma}
\end{align*}
and they discount the future with the factor $\beta \in (0,1)$. Each person has access to a bond that pays interest rate $r$. Their budget constraint can then be written as:
\begin{align*}
c + a'  = e^z + (1+r)a
\end{align*}
Let $\beta = 0.96$ and $\gamma = 1.0001$ for now. 

The interest rate $r$ is determined to clear the bond market. The bond is
available in zero net supply.


\section*{Exercícios}

\begin{enumerate}[wide, label = \arabic*.]

\item Let $\rho = 0.9$ and $\sigma = 0.01$. Use the Tauchen method to discretize the stochastic process in a Markov chain with 9 states. (Use 3 standard deviations for each side.)


\item Discretize the asset space using a grid and solve the individual's problem for each state variable. 


\item Find the stationary distribution $\pi(z,a)$ and use it to compute the aggregate savings in the economy. Find the equilibrium interest rate.


\item Suppose $\rho = .97$. Redo the analysis. How does the interest rate compare now? Explain.


\item Suppose $\gamma = 5$. Redo the analysis. How does the interest rate compare now? Explain.


\item Suppose $\sigma = .05$. Redo the analysis. How does the interest rate compare now? Explain. 


\item Ralate your results with Table 2 in Aiyagari (1994). 

\end{enumerate}


\end{document}

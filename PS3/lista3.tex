\documentclass[10pt]{article}
\usepackage[margin=2cm]{geometry}
\usepackage{lipsum}
\usepackage{amsmath}
\usepackage[inline]{enumitem}
\usepackage{verbatim}
\usepackage{fancyvrb}
\usepackage{pgfornament}
\usepackage{amssymb, tabularx, xcolor, nccmath}
\usepackage[utf8]{inputenc}
\usepackage{cancel}
\usepackage[portuguese]{babel}
\usepackage{bbm}
\usepackage{titlesec}
\usepackage{setspace, mathtools}
\usepackage{multirow}

\titleformat*{\section}{\sffamily\large\bfseries}
\titleformat*{\subsection}{\sffamily\normalsize\bfseries}
\makeatletter
\renewcommand*\env@matrix[1][\arraystretch]{%
  \edef\arraystretch{#1}%
  \hskip -\arraycolsep
  \let\@ifnextchar\new@ifnextchar
  \array{*\c@MaxMatrixCols c}}
\makeatother

\usepackage{hyperref}
\hypersetup{
    colorlinks=true,
    linkcolor=magenta,
    filecolor=magenta,      
    urlcolor=magenta,
}

\usepackage[T1]{fontenc}
\usepackage{ccfonts}
\renewcommand{\bfseries}{\fontfamily{cmbr}\fontseries{bx}\selectfont}
\DeclareMathAlphabet{\mathbf} {OT1}{cmbr}{bx}{n}
\DeclareMathAlphabet{\mathbold}{OML}{cmbrm}{b}{it}

\setlength{\parindent}{4em}
\setlength{\parskip}{1em}
\renewcommand{\baselinestretch}{1.2}


\def\changemargin#1#2{\list{}{\rightmargin#2\leftmargin#1}\item[]}
\let\endchangemargin=\endlist 



\DeclareMathOperator*{\argmin}{arg\,min}

\newcommand*{\QEDA}{\hfill\ensuremath{\blacksquare}}%
\newcommand*{\QEDB}{\hfill\ensuremath{\square}}%
\DeclareMathOperator*{\plim}{plim}

\setlength\parindent{0pt}

\usepackage{dsfont}

\newcommand\Z{\mathbf{Z}}
\newcommand\E{\mathbb{E}}

\newcommand\R{\mathbb{R}}
\newcommand\D{\mathbf{D}}
\newcommand\C{\mathbf{D}}
\newcommand\X{\mathbf{X}}
\newcommand\yy{\mathbf{y}}
\newcommand\ii{{\boldsymbol{\iota}}}
\renewcommand\O{{\boldsymbol{\Omega}}}
\newcommand\y{\mathbf{y}}
\newcommand\e{\mathbf{e}}
\renewcommand\u{\mathbf{u}}
\renewcommand\d{\mathbf{d}}
\newcommand\resid{\mathbf{u}}
\newcommand\0{\mathbf{0}}
\newcommand\betabold{\pmb{\beta}}
\newcommand\ones{\boldsymbol{\iota}}
\newcommand\epbold{\boldsymbol{\epsilon}}
\newcommand\varep{\boldsymbol{\varepsilon}}
\newcommand\x{\mathbf{x}}
\newcommand\z{\mathbf{z}}
\newcommand\Q{\mathbf{Q}}
\newcommand\I{\mathbf{I}}
\newcommand\w{{\mathbf{w}}}
\newcommand\wm{\bar{\mathbf{w}}}
\newcommand\M{{\mathbf{M}}}
\renewcommand\P{{\mathbf{P}}}
\newcommand\var{\operatorname{var}}
\newcommand\cov{\operatorname{cov}}
\renewcommand\b{\mathbf{b}}
\renewcommand\e{\mathbf{b}}


\usepackage{amssymb,graphicx}
\def\tallqed{\smash{\scalebox{.75}[1.025]{\color{blue!50!black}$\blacksquare$}}}

\tolerance=1
\emergencystretch=\maxdimen
\hyphenpenalty=10000
\hbadness=10000

\newcommand{\prnt}[1]{\ensuremath{\left(#1\right)}} %parentheses
\newcommand{\colch}[1]{\ensuremath{\left[#1\right]}} %square brackets
\newcommand{\chave}[1]{\ensuremath{\left\{#1\right\}}}  %curly brackets

\usepackage{stackengine}
\renewcommand\useanchorwidth{T}
\usepackage{graphicx}
\stackMath

\allowdisplaybreaks

\usepackage{MnSymbol}

\newcommand{\mytext}[1]% #1 = same as intertext
{&\parbox{0.94\textwidth}{\rule{0pt}{.5\baselineskip}\\
\textrm{#1}\\
\rule{0pt}{.5\baselineskip}}&\\}

\newcounter{exercise}
\newcounter{problem}[exercise]
\newcommand{\myitem}{\stepcounter{problem}\tag*{\alph{problem})}}

\usepackage{amsmath, amsthm, amssymb, amsfonts, enumitem, fancyhdr, color, comment, graphicx, environ, csquotes}


\newenvironment{problem}[2][Problem]{\begin{trivlist}
\item[\hskip \labelsep {\bfseries #1}\hskip \labelsep {\bfseries #2.}]}{\end{trivlist}}
\newenvironment{sol}
    {\\[1em] {\color{magenta}\text{Resposta.}}
    }
    {{\color{blue!50!black}\QEDA}}

\setlength{\parskip}{\baselineskip}%

\usepackage[framed,numbered,autolinebreaks,useliterate]{mcode}
\lstset{breakatwhitespace=false} %%<---this line added
\usepackage{listings}
\lstset{
    language=Matlab,
    escapeinside={\%*}{*)},
    breaklines=true,
    extendedchars=false,
    inputencoding=utf8
    }
    
\begin{document}

\stepcounter{exercise}
\newdimen\headerwidth


\begin{center}
  \framebox{
    \vbox{
      \headerwidth=\textwidth
      \vspace{1mm}
      \advance\headerwidth by -0.22in
      \hbox to \headerwidth {\it Doutorado em Economia - EPGE/FGV \hfill MDPEMF024 - Métodos Numéricos}
      \vspace{5mm}
      \hbox to \headerwidth {{\Large \hfill Lista \#3 \hfill}}
      \vspace{6mm}
      \hbox to \headerwidth {\hfill \today \hfill}
      \vspace{5mm}
      \hbox to \headerwidth {{\it Aluno: Rafael Vetromille  \hfill Professor: Cézar Santos / TA: Ana Paula Ruhe}}
      \vspace{1mm}
      }
    }
\end{center}

Para esta lista, você terá que solucionar o modelo de RBC usando diferentes técnicas numéricas. O modelo é bastante padrão. Aqui, darei uma breve descrição. Para mais detalhes, ver, por exemplo, Cooley e Prescott (1995).

\textbf{Preferências}

Os indivíduos têm preferências dadas por: 
\begin{align*}
U(C) = \E_0 \sum_{t=0}^\infty \beta^t u(C_t) \qquad \text{em que} \qquad 
u(C_t) = \frac{C_t^{1-\mu} - 1}{1 - \mu}
\end{align*}
e $\beta = 1/(1+\xi)$.

\textbf{Tecnologia}

Há uma firma representativa que se defronta com a seguinte função de produção:
\begin{align*}
Y_t = z_t F(K_t, N_t) = z_t K_t^\alpha N_t^{1-\alpha}, 
\end{align*}
em que $Y_t$ é o produto, $K_t$ é o estoque de capital, $N_t$ é o trabalho e $z_t$ é a produtividade total dos fatores (TFP), que é estocástica. O estoque de capital se deprecia a uma taxa $\delta$. 

Para $z_t$, assuma um processo AR(1) em logs tal que:
\begin{align*}
\log z_t = \rho \log z_{t-1} + \varepsilon_t \qquad \text{com} \qquad \varepsilon_t \sim N(0, \sigma^2)
\end{align*}\vspace{-0.8cm}

\textbf{Equilíbrio}

Note que o primeiro teorema do bem estar vale para essa economia. Assim, você pode resolver o problema do planejador central para encontrar a alocação.

\textbf{Calibração}

Precisamos de alguns valores para os parâmetros. Use $\beta = 0.987$, um valor padrão. O coeficiente de aversão relativa ao risco $\mu = 2$, também padrão. Para a função de produção, use $\alpha = 1/3$, o que implica uma razão entre renda de trabalho e renda de $2/3$, consistente com os dados. Use uma taxa de depreciação $\delta = 0.012$. Para o processo estocástico do choque de produtividade, use os valores de Cooley e Prescott (1995): $\rho = 0.95$ e $\sigma = 0.007$.

\newpage

\section*{Exercícios}

\begin{enumerate}[wide, label = \arabic*.]

\item Para resolver esta lista, utilizaremos métodos de projeção. Para este item, resolva o problema usando um método de projeção global. Em
particular, utilize polinômios de Chebyshev e o método da colocação
(\textit{collocation points}) para resolver o problema. Forneça evidências sobre sua solução: figuras da função valor e/ou função política, tempo de
execução, Euler errors, etc.
\begin{sol}

\end{sol}


\item Para este item, novamente utilize um método de projeção, mas, em vez
de um método espectral, use o método dos elementos finitos. Divida o
espaço de estado em diversos elementos. Para resolver este problema,
tente utilizar tanto o método da colocação quanto Galerkin\footnote{ Para utilizar o método de Galerkin, você precisará computar integrais numéricas. Para isso, você precisará utilizar quadraturas. Como referências, veja os slides de Grey Gordon (\href{https://sites.google.com/site/greygordon/teaching}{aqui}) ou o livro do Judd (1998).}. De novo, evidências!
\begin{sol}

\end{sol}

\end{enumerate}



\begin{thebibliography}{100} 

\bibitem{Cooley95} Cooley, T. F., \& Prescott, E. C. (1995). Economic growth and business cycles. Frontiers of Business Cycle Research / Thomas F. Cooley, Editor.

\bibitem{Judd88} Judd, K. L. (1998). Numerical methods in economics. Cambridge, Mass: MIT.

\end{thebibliography}


\end{document}
